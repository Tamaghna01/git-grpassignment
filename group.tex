\documentclass[12pt]{article}
\usepackage{geometry}
\geometry{a4paper, margin=1in}
\usepackage{fancyhdr}
\usepackage{titlesec}
\usepackage{hyperref}
\usepackage{graphicx}
\usepackage{amsmath}

% Title Page
\title{
    \vspace{0.2in}
    \Huge \textbf{Maulana Abul Kalam Azad University of Technology} \\
    \vspace{0.5in} % Reduced the vertical space
    \begin{center}
        \includegraphics[width=0.6\textwidth]{Pic.png} % Add your logo image here
    \end{center}
    \vspace{0.5in}
    \Huge\textbf{\textit{Lab Notebook}} \\
    \vspace{0.5in}
    \newpage
    \Large \textbf{\textit{GROUP MEMBERS}} \\
    \vspace{0.5in}
}
\date{}

\begin{document}
\maketitle

% Group Members Information Table
\begin{center}
    \begin{tabular}{|c| c | c | c |}
    \hline
    \textbf{SL NO}&\
    \textbf{Name} & \textbf{Roll Number} & \textbf{Department} \\
    \hline
     1 & TAMAGHNA GHOSH &30001223025 & BCA  \\
    \hline
     2& SRITI MAJUMDER & 30001223078 & BCA \\
    \hline
     3& SOUMYAJIT KHAN & 30001223019 & BCA   \\
    \hline
     4 & HIMADRI BERA & 30001223009 & BCA \\
    \hline
     5 & MOHIMA ISLAM & 30001223049 & BCA \\
    \hline
    \end{tabular}
\end{center}
\newpage

    
\begin{flushleft}
 \subsection*{INDEX}
  \begin{tabular}{|c|  c | c|c|c| }
    \hline
    \textbf{SL NO}&\
    \textbf{INDEX} & \textbf{DATE}&\textbf{ENTRY BY} &\textbf{PAGE NO} \\
    \hline
     1 & My First \LaTeX & 30/8/24&TAMAGHNA GHOSH & 4-7\\
    \hline
     2&Git Branching and Merging&1/9/24& TAMAGHNA GHOSH &8 \\
    \hline
     3&  MY CV & 06/09/24 & SRITI MAJUMDER & 10\\
    \hline
    4& GIT HUB &7/9/24 & SOUMYAJIT KHAN &11-12\\
     \hline
    5&About Latex & 8/9/24&HIMADRI BERA&13-14\\
    \hline
    6&MY FIRST LATEX DOCUMENT&10/09/24&MOHIMA ISLAM&15-18\\
    \hline
    7& ABOUT CV& 11/09/2024&MOHIMA ISLAM&19\\
     \end{tabular}
\end{flushleft}

\newpage

% Lab Notebook Entries
\subsection*{Lab Notebook Entry by Member 1}
\subsection*{Date: 08-09-2024}

\begin{flushright}
\textbf{Name:} TAMAGHNA GHOSH \\
\textbf{Semester:} 2nd \\
\textbf{Year:} I \\
\end{flushright}

\begin{center}
\Huge \textbf{ACKNOWLEDGEMENT: GIT}
\end{center}

I am really grateful to my professor for assigning such a wonderful project, which has helped us gain valuable work experience. I also appreciate my peers for choosing me as the group leader, and I hope to meet their expectations in this role.
\newpage

\subsection*{MY FIRST LATEX DOCUMENT}
\subsection*{Date: 30-08-2024}

\begin{center}
    \includegraphics[width=0.2\textwidth]{tamaghna.jpg}\\[1cm]
    
\end{center}

\section*{Introduction}
Hello !! My name is Tamaghna Ghosh. I am currently a 1st year undergrad at MAKAUT University doing my BCA.

\subsection*{Interests \& Hobbies}
My hobbies are drawing and a bit of singing. I love to paint and have a nigh interest towards graphic design. Some of my other hobies are :

\begin{itemize}
    \item Playing batminton
    \item hanging out with friends
        \begin{itemize}
            \item Usually catching up after classes
        \end{itemize}
    \item Doing Coding
\end{itemize}

\subsection*{Favorite Quotations}
\begin{enumerate}
    \item I think I can. I know I can. – Jennifer Wittwer
    \item Only I can change my life. No one can do it for me. – Carol Burnett
\end{enumerate}


\section*{2 Mathematics}

\subsection*{2.1 Mathematics and Me}
Mathematics has always been a very favourite subject of mine. It has always interested me and motivated me to believe in the fact that every problem is solveable.

\subsection*{2.2 Mathematical Notation}
I choose the number 2004 

\begin{enumerate}
    \item Superscripts, subscripts, and Greek letters
    \begin{itemize}
        \item \(2004\)
        \item \(\mu^{2004}\)
        \item \(\pi_{2004}\)
        \item \(2004^{\alpha}\)
        \item \(2004_m\)
        \item \(\tau_{2004}\)
        \item \(\log_{2004}\)
        \item \(\ln 2004\)
        \item \(0 < x \leq 2004\)
        \item \(\sum_{i=1}^{2004} x_i\)
    \end{itemize}
    \item Roots, fractions, and displaystyle
    \begin{itemize}
        \item \(\sqrt{2004}\)
        \item \(\sqrt[3]{2004}\)
        \item \(\frac{2004}{2}\)
        \item \(\frac{\sqrt{2004}}{2}\)
        \item \(\frac{1}{\sqrt{2004}}\)
    \end{itemize}
    \item Delimiters
    \begin{itemize}
        \item \(\left( \frac{2004}{2} \right)\)
        \item \(\left[ \frac{2004}{2} \right]\)
        \item \(\left\{ \frac{2004}{2} \right\}\)
    \end{itemize}
\end{enumerate}

\section*{Tables and Continuous Functions}
\begin{enumerate}
    \item[(a)] \(\lim_{x \to 3} \frac{x^2 - 9}{x - 3} = ?\)
    \item[(b)] \(\lim_{x \to -2} \frac{x^2 + x - 6}{x + 2} = ?\)
    \item[(c)] \(\lim_{h \to 0} \frac{(3+h)^2 - 9}{h} = ?\)
\end{enumerate}

\section*{Functions \& Formulas}
\begin{enumerate}
    \item[(a)] \(f(x) = x^3 - x\)
    \begin{enumerate}
        \item[i.] \(f'(x) = ?\)
        \item[ii.] \(f''(x) = ?\)
    \end{enumerate}
    \item[(b)] \(g(x) = \frac{1}{1+x^2}\)
    \begin{enumerate}
        \item[i.] \(g'(x) = ?\)
    \end{enumerate}
\end{enumerate}

\section*{Fundamental Theorem of Calculus, Part I}
If \(F\) is an antiderivative for \(f\) on an interval \(I\), then for every continuous function on \(I\),
\[ \int_a^b f(x)\, dx = F(b) - F(a). \]

\section*{Fundamental Theorem of Calculus, Part II}
If \(f\) is continuous on \([a,b]\), then the function defined by
\[ F(x) = \int_a^x f(t)\, dt \]
for all \(x\) in \([a,b]\) is continuous on \([a,b]\) and differentiable on \((a,b)\), and \(F'(x) = f(x)\).

\section*{Example}
Method vs use of F.T.C., part 2 in the step-by-step solution.
\[ \int_0^{1972} \frac{dx}{1+x^2} = [\arctan x]_0^{1972} = \ldots \]

\section*{Taylor Series}
Convergent geometric series since \(|r| < 1\).
\[ e^{-t/12} \sin t = \sum_{n=0}^\infty (-1)^n \frac{t^{n+1}}{(12)^n (n+1)!} \]
\newpage
\subsection*{Git Branching and Merging}
\subsection*{Date: 01-09-2024}
\title{Git Branching and Merging{} Document}
Assignment: Git Branching and Merging

Objective: Demonstrate proficiency in Git branching, merging, and conflict resolution.

Task step by step:

1. Create a new repository on GitHub called "git-advanced".
2. Clone the repository to your local machine.
3. Create a new branch called "feature-1" and switch to it.
4. Create a new file called "shared.txt" and add the following text:

This is a shared file.
Line 1: Original text.
Line 2: Original text.

5. Stage and commit the file with a meaningful message.
6. Push the branch to GitHub.
7. Create another branch called "feature-2" and switch to it.
8. Checkout the "shared.txt" file from the main branch (to ensure you're starting with the same file).
9. Modify the "shared.txt" file by changing the second line to:

Line 2: Modified text in feature-2.

10. Stage and commit the changes with a meaningful message.
11. Push the branch to GitHub.
12. Switch back to the "feature-1" branch.
13. Modify the "shared.txt" file by changing the second line to:

Line 2: Modified text in feature-1.

14. Stage and commit the changes with a meaningful message.
15. Push the branch to GitHub.
16. Merge "feature-1" into the main branch.
17. Merge "feature-2" into the main branch, which should introduce a conflict.
18. Resolve the conflict by editing the file and committing the resolution.
19. Push the updated main branch to GitHub.
20. Delete the "feature-1" and "feature-2" branches.

Deliverables:
You need to upload only ONE pdf file consists of the below things:
A screenshot of your GitHub repository showing the commit history and branching.
A screenshot of your local machine showing the Git log and conflict resolution.
A brief write-up of your experience with Git branching and merging.
Grading criteria:
Correct creation and management of branches.
Accurate merging and conflict resolution.
Clear and meaningful commit messages.
Proper deletion of branches.
\newpage


% Lab Notebook Entries
\subsection*{Lab Notebook Entry by Member 2}
\subsection*{Date: 08-09-2024}

\begin{flushright}
\textbf{Name:} SRITI MAJUMDER \\
\textbf{Semester:} 2nd \\
\textbf{Year:} I \\
\end{flushright}

\begin{center}
\Huge \textbf{ASSIGNMENT: GIT}
\end{center}

I am immensely grateful to my teacher, Ayan Sir, for assigning this remarkable group project, which has provided us with invaluable hands-on experience. I also want to extend my appreciation for the unwavering support and teamwork throughout this endeavor. Additionally, I have able to learn how to  resolved minor issues in my  code to enhance its functionality.


\newpage
\subsection*{MY CV}
\subsection*{Date: 06-09-2024}
% Lab Notebook Entries


\noindent
\begin{minipage}{0.3\textwidth}
    \vspace{0.5cm}
    \includegraphics[width=0.9\textwidth]{.jpg}  % Add your photo here
    
    \vspace{0.5cm}
    \colorbox{
        \begin{minipage}{\textwidth}
            \vspace{0.3cm}
            \textbf{PROFILE} \\
            Bachelor in Computer Application student. I consider myself a responsible and orderly person.\\
            I am looking forward to my first work experience.
            \vspace{0.3cm}
        \end{minipage}
    }
    
    \vspace{0.5cm}
    {
        \begin{minipage}{\textwidth}
            \vspace{0.3cm}
            \textbf{CONTACT ME} \\
            \textbf{Phone:} +91-9382875791 \\
            \textbf{Email:sritimajumder95@gmail.com} \\
            \textbf{Address:} Gangni, Badkulla, West Bengal, Nadia, 741121 \\
            \vspace{0.3cm}
        \end{minipage}
    }
\end{minipage}
\hfill
\begin{minipage}{0.65\textwidth}
    \section*{SRITI MAJUMDER}
    \textbf{Student}
    
    \section*{EDUCATION}
    \begin{itemize}
        \item 2022-23: West Bengal Council of Higher Secondary Education \\
        \textbf{H.S.} - 67.8\%
        \item 2020-21: West Bengal Board of Secondary Education \\
        \textbf{WBBSE} - 84.85\%
        \item 2023-24: Information Technology Application - 78\%
    \end{itemize}
    
    \section*{LANGUAGE}
    \begin{itemize}
        \item Native English
        \item Hindi
        \item Bengali
    \end{itemize}
    
    \section*{COMPUTER SKILLS}
    \begin{itemize}
        \item Well versed in MS Excel, MS Word, MS PowerPoint
        \item DBMS (Visual FoxPro)
        \item Spreadsheet
        \item Slide Presentation
    \end{itemize}
    
    \section*{PERSONAL ABILITY}
    \begin{itemize}
        \item Good Communication Skills
        \item Able to Work Independently and With The Team
        \item Hardworking, Goal-Oriented, Adaptive, Simple, and Supportive
    \end{itemize}
    
\end{minipage}



\subsection*{Lab Notebook Entry by Member 3}
\subsection*{Date: 08-09-2024}

\begin{flushright}
\textbf{Name:} SOUMYAJIT KHAN \\
\textbf{Semester:} 2nd \\
\textbf{Year:} I \\
\end{flushright}

\begin{center}
\Huge \textbf{ASSIGNMENT: GIT}
\end{center}

I am deeply thankful to my teacher, Ayan Sir, for giving us this exceptional group project. It has been an incredible opportunity for hands-on learning and skill development. I also want to express my gratitude for the consistent support and collaborative effort throughout this journey. Furthermore, I've gained valuable experience in troubleshooting and refining my code to improve its functionality.

\newpage
\subsection*{GIT HUB}
\subsection*{Date: 07-09-2024}
GitHub
GitHub is a web-based platform that allows developers to host, share, and
collaborate on software projects. It provides a version control system
powered by Git, enabling teams to track changes, manage code repositories,
and work together seamlessly across different locations. GitHub supports
collaborative development through features like pull requests, issues, and
project boards, making it essential for open-source projects and professional
software development. Additionally, it integrates with various development
tools, enhancing productivity and streamlining the software development
lifecycle.
Installation
Installing GitHub Desktop is a straightforward process that enhances your
workflow by providing a user-friendly interface for managing repositories.
To begin, download the installer from the [official GitHub Desktop
website](https://desktop.github.com/) for your operating system—Windows
or macOS. After downloading, run the installer and follow the on-screen
instructions to complete the setup. Once installed, launch the application
and sign in with your GitHub credentials, or create a new account if needed.
GitHub Desktop simplifies the process of cloning repositories, making
commits, and managing branches, making it an invaluable tool for
developers of all skill levels. For Linux users, alternative methods like using
Wine or other Git clients are available.





\newpage
% Lab Notebook Entries
\subsection*{Lab Notebook Entry by Member 4}
\subsection*{Date: 08-09-2024}

\begin{flushright}
\textbf{Name:} HIMADRI BERA \\
\textbf{Semester:} 2nd \\
\textbf{Year:} I \\
\end{flushright}

\begin{center}
\Huge \textbf{ASSIGNMENT: GIT}
\end{center}

I'm incredibly thankful to my professor for assigning such a fantastic project, which has given us invaluable hands-on experience. I'm also grateful to my teammates for their collaboration, and I hope to contribute meaningfully to the team's success.

\newpage
\subsection*{About Latex}
\subsection*{Date: 08-09-2024}
LaTeX
LaTeX is a powerful document preparation system widely used in academia,
research, and professional publishing. It is based on the TeX typesetting
system and allows users to create high-quality documents, including articles,
books, reports, and presentations. LaTeX is particularly popular in fields
that require complex formatting, such as mathematics, physics, and
computer science, due to its superior handling of formulas, citations, and
references. LaTeX separates content from formatting, allowing writers to
focus on the content itself while the system ensures consistent and
professional typesetting.
Usage
Using LaTeX begins with writing plain text in a source file, which contains
both the content of the document and LaTeX commands for formatting.
LaTeX files typically have a .tex extension. To compile a LaTeX document,
a LaTeX distribution (such as TeXLive or MikTeX) is used, converting the
source file into a well-formatted PDF output.
The system excels at managing references, creating bibliographies, and
producing structured documents such as theses and dissertations. Users
define sections, equations, tables, and figures using simple commands, and
LaTeX ensures they are consistently formatted throughout the document.
Moreover, LaTeX supports various packages that extend its functionality,
such as amsmath for mathematical notation and graphicx for including
images. Many integrated development environments (IDEs) like Overleaf,
TeXShop, and Texmaker simplify the process of writing and compiling
LaTeX documents.
In summary, LaTeX is a versatile tool that ensures professional-quality
typesetting while providing flexibility and control over complex document
formatting. It is invaluable for writers in technical fields or anyone looking
to produce polished, well-organized documents.
\newpage


% Lab Notebook Entries
\subsection*{Lab Notebook Entry by Member 5}
\subsection*{Date: 08-09-2024}

\begin{flushright}
\textbf{Name:} MOHIMA ISLAM \\
\textbf{Semester:} 2nd \\
\textbf{Year:} I \\
\end{flushright}

\begin{center}
\Huge \textbf{ASSIGNMENT: GIT}
\end{center}

I express my gratitude towards my teacher for giving the opportunity to do this assignment. This project has enhanced my experience, skill and also gave me a chance to do teamwork. I am also thankful to my teammates for their support and I'm praying for the best result. Thank you.
\newpage
\section{MY FIRST LATEX DOCUMENT}
\subsection*{Date: 10-09-2024}
 \begin{center}
        \includegraphics[width=0.5\linewidth]{mohima.jpg}
     \end{center}
\section{Introduction}
\subsection{About Me}
\paragraph{I am Mohima Islam. I'm a BCA student of Maulana Abul Kalam Azad University of Technology. I live in Barasat, North 24 Parganas, West Bengal. My hobby is to read story books. I also like travelling very much. I have visited some popular places in my state like The Indian Museum, Victoria Memorial, Digha, Darjeeling etc. I love my family very much. }
\subsection{Hobbies and Interests}
\begin{itemize}
    \item \textbf{Thing 1: }I love to read story books. I have a huge collection of bengali and english story books. I love the horror, thriller, mysterious, detective and science fiction stories. I also like novels and poems.
    \item \textbf{Thing 2: }I have interest in Roman and Greek mythology. I have read many books on this and also seen many videos. I am eager to know many more about the Greek and Roman mythology. 
\end{itemize}
\subsection{Favourite Quotation}
\begin{enumerate} 
    \item "Seek knowledge from the cradle to the grave"\\
    - Hazrat Muhammad.
    \item "What does a person desire --- progress or happiness? What is the use of progress if it does not bring happiness?"\\
    - Bibhutibhushan Bandyopadhyay.
    \end{enumerate}
    \section{Mathematics}

\subsection{Mathematics and Me}
I like mathematics. When I was in school mathematics used to be my second favourite subject. I really enjoy calculus part of mathematics. The tricks of application attracts me the most. But sometimes it also felt challenging to me when after a long calculation the answer came wrong.
\subsection{Mathematical Notation}
\begin{enumerate}
    \item Superscripts, subscripts, and Greek letters
    \begin{enumerate}
        \item $12^{36}$
        \item $1^{2^{3^6}}$
        \item $^{12_{36}}$
        \item $^{1236}{\pi}$
        \item $\cos \theta$
        \item $\tan^{-1}(1.236)$
        \item $\log_{12} 36$
        \item $\ln 1236$
        \item $e^{1.236}$
        \item $0 < x \leq 1236$
        \item $y \geq 1236$
    \end{enumerate}
    \item Roots, fractions, and displaystyle
    \begin{enumerate}
        \item $\sqrt{1236}$
        \item $12\sqrt{36}$
        \item Normal: $\frac{12}{36}$, Displaystyle: $\displaystyle\frac{12}{36}$
        \item Normal: $\frac{1}{2+3}$, Displaystyle: $\displaystyle\frac{1}{2+3}$
        \item Normal: $\sqrt{\frac{12}{36}}$, Displaystyle: $\displaystyle\sqrt{\frac{12}{36}}$
    \end{enumerate}

    \item Delimiters
    \begin{enumerate}
        \item Display math mode: $\left(\frac{1}{2+36}\right)$
        \item Display math mode: $\left|\frac{12-3}{6}\right|$
    \end{enumerate}

    \item Tables and equation arrays
    \begin{enumerate}
        \item 
        \[
        \begin{array}{c|c|c|c|c}
        x & 1 & 2 & 3 & 4 \\
        \hline
        f(x) & 1 & 2 & 3 & 6 \\
        \end{array}
        \]

  \item 
        \begin{align}
        1 + 2 - 3 \times 6 &= x \tag{1}\\
        1 + 2 - 18 &= x \tag{2}\\
        3 - 18 &= x \tag{3}\\
        x &= -15 \tag{4}
        \end{align}
    \end{enumerate}
\end{enumerate}

\subsection{Functions \& Formulas}
\begin{enumerate}
    \item The quadratic formula:
    \[
    x = \frac{-b \pm \sqrt{b^2 - 4ac}}{2a}
    \]
    
    \item The function $f(x) = \left(\frac{x+1}{2}\right)^2 - \frac{3}{6}$ has domain $D_f: (-\infty, \infty)$ and range $R_f: \left[\frac{-3}{6}, \infty\right)$.
    
    \item Definition of a Derivative: 
    \[
    \lim_{h \to 0} \frac{f(x+h) - f(x)}{h} = f'(x)
    \]
    
    \item Chain Rule: $[f(g(x))]' = f'(g(x)) \cdot g'(x)$
    
    \item $\frac{d^2y}{dx^2} = f''(x)$
    
    \item $\int \sec^2 x dx = \tan x + C$
 \item $\int e^{2x} dx = \frac{1}{2}e^{2x} + C$
    
    \item Fundamental Theorem of Calculus, Part 1:
    \[
    \int_{a}^{b} f'(x) dx = f(b) - f(a)
    \]
    
    \item Fundamental Theorem of Calculus, Part 2:
    \[
    \frac{d}{dx} \int_{a}^{g(x)} f(t) dt = f(g(x)) \cdot g'(x)
    \]
    
    \item Euler's Method: $y_1 = y_0 + hF(x_0, y_0)$ where $h$ is the step size, and $F(x, y) = \frac{dy}{dx}$
    
    \item $a_n = \left\{1236, \frac{1236}{2}, \frac{1236}{2^2}, \frac{1236}{2^3}, \ldots, \frac{1236}{2^n}\right\}$ represents a geometric sequence.
    
    \item $S_n = \sum_{n=1}^{\infty} \frac{1236}{2^n}$ is a convergent geometric series since $\left|\frac{1}{2}\right| < 1$.
    
    \item Taylor Series:
    \[
    \sum_{n=0}^{\infty} \frac{f^{(n)}(c)}{n!} (x - c)^n
    \]
    
    \item Velocity Vector: $\vec{v}(t) = x'(t)\vec{i} + y'(t)\vec{j} = \left\langle \frac{dx}{dt}, \frac{dy}{dt} \right\rangle$
    
    \item Area of Polar Curve: $A = \frac{1}{2} \int_{\alpha}^{\beta} r^2 d\theta$
\end{enumerate}
\newpage
\subsection*{MY CV}
\subsection*{Date: 11-09-2024}
\begin{center}
    {\LARGE \textbf{Mohima Islam}} \\
    \vspace{2mm}
    \href{mailto:mohimaislam1236@gmail.com}{mohimaislam1236@gmail.com} \\
    \href{https://www.linkedin.com/in/Mohima Islam}{linkedin.com/in/Mohima Islam} \\
    \href{https://github.com/Mohima-Islam}{github.com/Mohima-Islam} \\
    \vspace{2mm}
    \hrule
\end{center}
\section*{About Me}
\textbf{My name is Mohima Islam. I am a first year student of Maulana Abul Kalam Azad University of Technology. I am pursuing BCA. I like technologies very much so I want to make my career in this field. My dream is to be a Web Developer. I want to serve my country as possible as I can from my side. }
\section*{Education}
\begin{itemize}[leftmargin=0.5cm]
    \item \textbf{Higher Secondary}, Marks Obtained: 82.4 \\
    \textit{Baduria LMS High School, Baduria, India} \\
    2023
    \item \textbf{Madhyamik}, Marks Obtained: 96\\
    \textit{Kulia Ramchandrapur Union High School, Kulia}\\
    2021
\end{itemize}
\section*{Computer Skills}
\begin{itemize}[leftmargin=0.5cm]
    \item Well versed in \textbf{MS Word, Ms Exel, MS PowerPoint}.
    \item \textbf{DBMS}(Basic)
    \item Programming languages: \textbf{ HTML, C, Python}.
\end{itemize}

\section*{Personal Skills}
\begin{itemize}[leftmargin=0.5cm]
    \item Good communication Skill. \\
    \item Able to work independently or with team.\\
    \item Highly energetic and self-motivated.
\end{itemize}

\section*{Languages}
\begin{itemize}[leftmargin=0.5cm]
    \item \textbf{Bengali}- Mother Tongue
    \item \textbf{English}- Frequent
    \item \textbf{Hindi}- Basic
\end{itemize}

\section*{Hobbies}
\begin{itemize}[leftmargin=0.5cm]
    \item Reading 
    \item Travelling
\end{itemize}
\end{document}